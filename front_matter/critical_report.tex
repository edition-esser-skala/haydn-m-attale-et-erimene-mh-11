\documentclass[tocstyle=ref-genre]{ees}

\begin{document}

\eesTitlePage

\eesCriticalReport{
  – & – & –  & Lyrics are hardly readable in \A1 and therefore may contain errors. \\
  – & – & bc & All bass figures have been added by the editor. \\
}

\eesToc{
\begin{movement}{inimica}
  \voice[Basso]
  Inimica mihi semper sydera,\\
  meum quae mihi amorem rapitis,\\
  ubi nunc mea quora gaudia,\\
  ubi lasso, ubi deponam cura senecta.
\end{movement}

\begin{movement}{vixit}
  \voice[Basso]
  Vixit, heu vixit Eumenes,\\
  quo me reductum video,\\
  iam mortem nunc non timeo,\\
  vitae dum absunt solatia.\\[1ex]
  Nunc ergo, alme Iupiter,\\
  trisulca qua te fulrixa\\
  in caput istud detona,\\
  mors mihi est laetita.
\end{movement}

\begin{movement}{pantaleon}
  \voice[Tenore]
  Pantaleon!\\
  Omnem absterge animo moerorem\\
  quod perditum doles praesentem tibi habes.\\
  Sed quid video, Numina!\\
  Meos recusat Pantaleon affatus\\
  quod animo tuo de me recursat usque dubium.
\end{movement}

\begin{movement}{aliquam}
  \voice[Tenore]
  Si aliquam mei\\
  tenes imaginem,\\
  hos vultus aspice,\\
  cor tuum consule,\\
  tum nega me Eumenem.\\[1ex]
  Ah! morari derine\\
  en quem luges,\\
  nunc salvum habes\\
  amare me iterum incipe.
\end{movement}

\begin{movement}{triumphate}
  \voice[Alto]
  Io triumphate socii,\\
  janne regna certa manent,\\
  Eumenem fratrem meum,\\
  quem unicum regni hostem timui,\\
  per laedium perdere constitui.
\end{movement}

\begin{movement}{quambonum}
  \voice[Alto]
  Quam bonum regnare,\\
  quam laetum videre\\
  pro meis, dum gentes,\\
  dum regna, dum duces\\
  obsequiis certant.\\[1ex]
  Si throni regnantes\\
  non sustinet fratres,\\
  non moror fratre perdere\\
  thronum ut possim scandere\\
  sceptra ut maneant.
\end{movement}

\begin{movement}{applaudo}
  \voice[Basso]
  Applaudo tibi Attale,
  per me consendis solium
  a fratro a morte regnum obtines.
  Felix utere quod tibi datum imperium,
  amplifica vires, rebelles cohibe
  et per domitos major resurge populos.
\end{movement}

% \begin{movement}{}
%   \voice[]
% \end{movement}

% \begin{movement}{}
%   \voice[]
% \end{movement}

% \begin{movement}{}
%   \voice[]
% \end{movement}

% \begin{movement}{}
%   \voice[]
% \end{movement}
\eesScore

\end{document}
