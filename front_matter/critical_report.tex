\documentclass[tocstyle=ref-genre]{ees}

\begin{document}

\eesTitlePage

\eesCriticalReport{
  –  & –   & –    & Lyrics are hardly readable in \A1 and therefore may contain errors. \\
     & –   & bc   & All bass figures have been added by the editor. \\
  8  & 122 & vl 1 & 1st note in \A1 illegible due to manuscript damage \\
  11 & 14  & T    & 2nd \halfNote\ in \A1: e′4–e′4 \\
     & 26  & T    & 4th \eighthNote\ in \A1: b′8 \\
}

\eesToc{
\begin{movement}{inimica}
  \voice[Basso]
  Inimica mihi semper sydera,\\
  meum quae mihi amorem rapitis,\\
  ubi nunc mea quora gaudia,\\
  ubi lasso, ubi deponam cura senecta.
\end{movement}

\begin{movement}{vixit}
  \voice[Basso]
  Vixit, heu vixit Eumenes,\\
  quo me reductum video,\\
  iam mortem nunc non timeo,\\
  vitae dum absunt solatia.\\[1ex]
  Nunc ergo, alme Iupiter,\\
  trisulca qua te fulrixa\\
  in caput istud detona,\\
  mors mihi est laetita.
\end{movement}

\begin{movement}{pantaleon}
  \voice[Tenore]
  Pantaleon!\\
  Omnem absterge animo moerorem\\
  quod perditum doles praesentem tibi habes.\\
  Sed quid video, Numina!\\
  Meos recusat Pantaleon affatus\\
  quod animo tuo de me recursat usque dubium.
\end{movement}

\begin{movement}{aliquam}
  \voice[Tenore]
  Si aliquam mei\\
  tenes imaginem,\\
  hos vultus aspice,\\
  cor tuum consule,\\
  tum nega me Eumenem.\\[1ex]
  Ah! morari derine\\
  en quem luges,\\
  nunc salvum habes\\
  amare me iterum incipe.
\end{movement}

\begin{movement}{triumphate}
  \voice[Alto]
  Io triumphate socii,\\
  janne regna certa manent,\\
  Eumenem fratrem meum,\\
  quem unicum regni hostem timui,\\
  per laedium perdere constitui.
\end{movement}

\begin{movement}{quambonum}
  \voice[Alto]
  Quam bonum regnare,\\
  quam laetum videre\\
  pro meis, dum gentes,\\
  dum regna, dum duces\\
  obsequiis certant.\\[1ex]
  Si throni regnantes\\
  non sustinet fratres,\\
  non moror fratre perdere\\
  thronum ut possim scandere\\
  sceptra ut maneant.
\end{movement}

\begin{movement}{applaudo}
  \voice[Basso]
  Applaudo tibi Attale,\\
  per me consendis solium\\
  a fratro a morte regnum obtines.\\
  Felix utere quod tibi datum imperium,\\
  amplifica vires, rebelles cohibe\\
  et per domitos major resurge populos.
\end{movement}

\begin{movement}{felicem}
  \voice[Basso]
  Felicem te principem,\\
  nunc te jam Pergama,\\
  nunc subdita flumina\\
  adorant regem.\\[1ex]
  I, felix impera\\
  dum te ut caesarem\\
  meritis jam gravem\\
  suscipiant Numina.
\end{movement}

\begin{movement}{nunctandem}
  \voice[Alto]
  Nunc tandem tentabo ultima altum\\
  in viscera defigam fratri mueronem.

  \voice[Basso]
  Ah! Attale melius consule vita\\
  en me fratri defensorem.

  \voice[Tenore]
  Numina quas in angustias redigorant\\
  throni aut ego perdor.

  \voice[Basso]
  Ah, desine tandem in fratre saevire.

  \voice[Tenore]
  Ah frater

  \voice[Alto]
  os comprime.
\end{movement}

\begin{movement}{morere}
  \voice[Alto, Tenore, Basso]
  Morere impie,\\
  vim cohibe perfide,\\
  ingrate superbe\\
  inique tyranne,\\
  ah, pacem nunc peto\\
  hanc ego detrecto\\
  proh dolor, proh poena,\\
  proh furor, proh ira,\\
  quis unquam cum hoste,\\
  quis unquam cum fratre\\
  ita certavit.\\[1ex]
  Ast vel loqui concede,\\
  quid vellis impie\\
  propone proh Dii,\\
  proh astra,\\
  quis unquam vel hostem\\
  ita amavit,\\
  unquam vel fratrem\\
  ita amavit.
\end{movement}

\begin{movement}{populi}
  \voice[Coro]
  Io triumphate populi,\\
  plausus date subditi,\\
  sceptrum quidem cessit Attalo,\\
  regio caret Eumenes honore.\\[1ex]
  Nunc duos fratres colitis at Reges,\\
  ubi unus vos imperio,\\
  alter regit amore.
\end{movement}
\eesScore

\end{document}
